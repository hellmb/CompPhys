\documentclass[12pt]{article}
\usepackage[utf8]{inputenc}
\usepackage{graphicx}
\usepackage{listings}
\usepackage{amsmath}
\usepackage{tensor}
\usepackage{subfigure}
\usepackage{float}
\usepackage{natbib}
\usepackage{url}
\usepackage{color}
\usepackage{hyperref}

%Include packages to have python(or other prog. languages) scripts included in the pdf-file without broken lines

\definecolor{codegreen}{rgb}{0,0.6,0}
\definecolor{codegray}{rgb}{0.5,0.5,0.5}
\definecolor{codepurple}{rgb}{0.58,0,0.82}
\definecolor{backcolour}{rgb}{0.95,0.95,0.92}
\lstdefinestyle{mystyle}{
    backgroundcolor=\color{backcolour},
    commentstyle=\color{codegreen},
    keywordstyle=\color{magenta},
    numberstyle=\tiny\color{codegray},
    stringstyle=\color{codepurple},
    basicstyle=\footnotesize,
    breakatwhitespace=false,
    breaklines=true,
    captionpos=b,
    keepspaces=true,
    numbers=left,
    numbersep=5pt,
    showspaces=false,
    showstringspaces=false,
    showtabs=false,
tabsize=2}
\lstset{style=mystyle}

%Making vectors bold instead of putting arrow on it
\renewcommand{\vec}[1]{\mathbf{#1}}

% Roman numbers
\newcommand{\RNum}[1]{\uppercase\expandafter{\romannumeral #1\relax}}

\begin{document}

\begin{titlepage}

\newcommand{\HRule}{\rule{\linewidth}{0.5mm}} 
\center

\textsc{\LARGE University of Oslo}\\[1.5cm] 
\textsc{\Large FYS4150/FYS3150}\\[0.3cm] 
\textsc{\large Helle Bakke, Kristine Moseid \& Helene Aune}\\[0.5cm] 

\begin{minipage}{0.4\textwidth}
\end{minipage}\\[1cm]

\HRule \\[0.4cm]
{ \huge \bfseries A Planetary Cluster of Differential Equations}\\[0.4cm] 
\HRule \\[1.5cm]
 
\begin{minipage}{0.4\textwidth}
\end{minipage}\\[8cm]


{\large \today}\\[3cm] 
\vfill 

\end{titlepage}

\newpage
\tableofcontents

\newpage


\begin{abstract}
\noindent Put abstract here
\end{abstract}

\section{Introduction}

\noindent The aim of this project was to develop a code for simulating the solar system. We used the velocity Verlet algorithm to solve coupled ordinary differential equations. In the development process, we set up $48$ coupled differential equations, illustrating all planets in the x-, y- and z-plane.\\

\noindent Initial conditions were obtained through the NASA website \url{http://ssd.jpl.nasa.gov/horizons.cgi#top}.

\section{Method}

\noindent For simplicity, we started with a system consisting of the Sun and Earth. We assumed that mass units could be obtained by using the fact that Earth's orbit is almost circular around the Sun. Using Newton's law for circular motion, we obtained the expression of force
\begin{equation}
F_G = \frac{M_E v^2}{r} = \frac{GM_{\odot}M_E}{r^2},
\end{equation}
\noindent where we used that
\begin{align*}
v^2r = GM_{\odot} = \frac{4 \pi^2 (AU)^3}{(yr)^2}
\end{align*}
\noindent We wanted to make our solar system 3-dimensional, and obtained the expressions for the derivatives.
\begin{align*}
\frac{d^2x}{dt^2} = \frac{F_{G,x}}{M_E}, \quad \frac{d^2y}{dt^2} = \frac{F_{G,y}}{M_E}, \quad \frac{d^2z}{dt^2} = \frac{F_{G,z}}{M_E}
\end{align*}
\noindent We used that $r = \sqrt{x^2 + y^2 + z^2}$, and obtained the equations
\begin{align*}
F_{G,x} = -\frac{GM_{\odot}M_E}{r^3}x, \quad F_{G,y} = -\frac{GM_{\odot}M_E}{r^3}y, \quad F_{G,z} = -\frac{GM_{\odot}M_E}{r^3}z
\end{align*}
\begin{equation} \label{set_eq2}
\begin{split}
\frac{dx}{dt} = v_x \rightarrow \frac{d^2x}{dt^2} = \frac{dv_x}{dt} = \dot{v_x} = \frac{F_{G,x}}{M_E} = -\frac{GM_{\odot}}{r^3}x\\
\frac{dv_y}{dt} = \dot{v_y} = \frac{F_{G,y}}{M_E} = -\frac{GM_{\odot}}{r^3}y\\
\frac{dv_z}{dt} = \dot{v_z} = \frac{F_{G,z}}{M_E} = -\frac{GM_{\odot}}{r^3}z\\
\end{split}
\end{equation}

\subsection{Euler's forward algorithm}

\noindent We wanted to set up Euler's forward algorithm for solving equation \ref{set_eq2}. For simplicity, we showed this for the x-direction, and copied the method for the two other directions.
\begin{align*}
x = x_i, \quad v_x = v_{x,i}, \quad t = t_i = t_0 + ih
\end{align*}
\noindent We set $h = \frac{t_F - t_0}{N}$, where $t_F$ was the final time, $t_0$ was the initial time, and $N$ was the number of integration points. By Taylor expanding both $x_i$ and $v_{x,i}$ we obtained the forward Euler algorithm.
\begin{equation} \label{taylor_pos}
x_{i+1} = x(t_{i+1}) = x_i + hx_i' + \frac{h^2}{2!}x_i'' + O(h^3)
\end{equation}
\begin{equation} \label{taylor_vel}
v_{x, i+1} = v_{x,i} + hv_{x,i}' + \frac{h^2}{2!}v_{x,i}'' + O(h^3)
\end{equation}

\noindent We truncated after the first derivative, and the algorithms looked like
\begin{equation} \label{euler_x}
\begin{split}
x_{i+1} &= x_i + hv_{x,i} + O(h^2)\\
v_{x, i+1} &= v_{x,i} + hv_{x,i}' + O(h^2)\\ 
&= v_{x,i} - h\frac{4 \pi^2}{r_i^3}x_i + O(h^2),
\end{split}
\end{equation}
\noindent remembering that $r_i = \sqrt{x_i^2 + y_i^2 + z_i^2}$. We did the same procedure for the y- and z-direction, and obtained the rest of the algorithms for solving the forward Euler method.
\begin{equation} \label{euler_y}
\begin{split}
y_{i+1} &= y_i + hv_{y,i} + O(h^2)\\
v_{y, i+1} &= v_{y,i} + hv_{y,i}' + O(h^2)\\ 
&= v_{x,i} - h\frac{4 \pi^2}{r_i^3}y_i + O(h^2)
\end{split}
\end{equation}
\begin{equation} \label{euler_z}
\begin{split}
z_{i+1} &= z_i + hv_{z,i} + O(h^2)\\
v_{z, i+1} &= v_{z,i} + hv_{z,i}' + O(h^2)\\
&= v_{x,i} - h\frac{4 \pi^2}{r_i^3}z_i + O(h^2)
\end{split}
\end{equation}

\subsection{Velocity Verlet algorithm}

\noindent Verlet integration is a numerical method used to integrate Newton's equation of motion \textbf{add reference to wiki}. The velocity Verlet method is a more commonly used algorithm, where velocity and position are calculated at the same value of the time variable. \\

\noindent As with Euler's forward algorithm, we started by performing Taylor expansions, where the expansion of position went as equation \ref{taylor_pos}. Instead of Taylor expanding the velocity, we used an expansion of the acceleration.
\begin{equation} \label{verlet_acc}
v_{x,i+1}' = v_{x,i}' + hv_{x,i}'' + O(h^2)
\end{equation}
\noindent We set $hv_{x,i}'' \simeq v_{x,i+1}' - v_{x,i}'$, and substituted this into \ref{verlet_acc} to obtain the algorithm for velocity. Finally, we had algorithms for both position and velocity.
\begin{equation} \label{verlet_x}
\begin{split}
x_{i+1} &= x_i + hv_{x,i} + \frac{h^2}{2} + O(h^3)\\
v_{x,i+1} &= v_{x,i} + \frac{h}{2}(v_{x,i+1}' + v_{x,i}') + O(h^3)\\
&= v_{x,i} - \frac{4 \pi^2 h}{2} \Big(\frac{x_{i+1}}{r_{i+1}^3} +\frac{x_i}{r_i^3} \Big) + O(h^3)
\end{split}
\end{equation}

\noindent By applying this in the remaining directions, we finally obtained all algorithms for solving the velocity Verlet method.
\begin{equation} \label{verlet_y}
\begin{split}
y_{i+1} &= y_i + hv_{y,i} + \frac{h^2}{2} + O(h^3)\\
v_{y,i+1} &= v_{y,i} + \frac{h}{2}(v_{y,i+1}' + v_{y,i}') + O(h^3)\\
&= v_{y,i} - \frac{4 \pi^2 h}{2} \Big(\frac{y_{i+1}}{r_{i+1}^3} +\frac{y_i}{r_i^3} \Big) + O(h^3)
\end{split} 
\end{equation}
\begin{equation} \label{verlet_z}
\begin{split}
z_{i+1} &= z_i + hv_{z,i} + \frac{h^2}{2} + O(h^3)\\
v_{z,i+1} &= v_{z,i} + \frac{h}{2}(v_{z,i+1}' + v_{z,i}') + O(h^3)\\
&= v_{z,i} - \frac{4 \pi^2 h}{2} \Big(\frac{z_{i+1}}{r_{i+1}^3} +\frac{z_i}{r_i^3} \Big) + O(h^3)
\end{split} 
\end{equation}



\addcontentsline{toc}{section}{References}
%\nocite{*}
%\bibliographystyle{plain}
%\bibliography{put bibtex-file here}

%https://en.wikipedia.org/wiki/Verlet_integration

\end{document}