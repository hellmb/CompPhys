\documentclass[12pt]{article}
\usepackage[utf8]{inputenc}
\usepackage{graphicx}
\usepackage{listings}
\usepackage{amsmath}
\usepackage{tensor}
\usepackage{subfigure}
\usepackage{float}
\usepackage{natbib}
\usepackage{url}
%Include packages to have python(or other prog. languages) scripts included in the pdf-file without broken lines
\usepackage{color}
\definecolor{codegreen}{rgb}{0,0.6,0}
\definecolor{codegray}{rgb}{0.5,0.5,0.5}
\definecolor{codepurple}{rgb}{0.58,0,0.82}
\definecolor{backcolour}{rgb}{0.95,0.95,0.92}
\lstdefinestyle{mystyle}{
    backgroundcolor=\color{backcolour},
    commentstyle=\color{codegreen},
    keywordstyle=\color{magenta},
    numberstyle=\tiny\color{codegray},
    stringstyle=\color{codepurple},
    basicstyle=\footnotesize,
    breakatwhitespace=false,
    breaklines=true,
    captionpos=b,
    keepspaces=true,
    numbers=left,
    numbersep=5pt,
    showspaces=false,
    showstringspaces=false,
    showtabs=false,
tabsize=2}
\lstset{style=mystyle}
%Making vectors bold instead of putting arrow on it
\renewcommand{\vec}[1]{\mathbf{#1}}

\begin{document}

\begin{titlepage}

\newcommand{\HRule}{\rule{\linewidth}{0.5mm}}
\center

\textsc{\LARGE University of Oslo}\\[1.5cm]
\textsc{\Large FYS4150}\\[0.5cm]
\textsc{\large Kristine Moseid, Helene Aune og Helle Bakke}\\[0.5cm]

\begin{minipage}{0.4\textwidth}
\end{minipage}\\[1cm]

\HRule \\[0.4cm]
{ \huge \bfseries Introduction to numerical projects}\\[0.4cm]
\HRule \\[1.5cm]

\begin{minipage}{0.4\textwidth}
\end{minipage}\\[8cm]


{\large \today}\\[3cm]
\vfill

\end{titlepage}

\newpage
\tableofcontents

\newpage

\section{Abstract}

\begin{itemize}
\item Tease the reader
\item Write last
\end{itemize}

\section{Introduction}

\begin{itemize}
\item Motivate the reader
\item What have we done
\item Structure of report
\end{itemize}

\noindent The aim of this project is to solve the one-dimensional Poisson equation with Dirichlet boundary conditions by rewriting it as a set of linear equations. We will be solving the equation
\begin{align*}
\frac{d^2\phi}{dr^r} = -4 \pi r \rho(r)
\end{align*}
\noindent By letting $\phi \rightarrow u$ and $r \rightarrow x$ it is simplified to
\begin{align*}
-u^{\prime \prime}(x) = f(x), \quad x \in (0,1), \quad  u(0) = u(1) = 0
\end{align*} 
where we define the discretized approximation to $u$ as $v_i$ with grid point $x_i = ih$ in the interval from $x_0$ to $x_{n+1} = 1$, and the step length as $h = 1/(n+1)$. \\

\noindent By doing this we will be able to create algorithms for solving the tridiagonal matrix problem, and find out how efficient this is compared to other matrix elimination methods. 


\section{Methods}

\begin{itemize}
\item Describe methods and algorithms
\item Explain
\item Calculations to demonstrate the code, verify results(benchmarks)
\end{itemize}

\subsection{Tridiagonal matrix}

\noindent With the bounadry condition $v_0 = v_{n+1} = 0$, the approximation of the second derivative of $u$ was written as 

\begin{align*}
- \frac{v_{i+1} + v_{i-1} - 2v_i}{h^2} = f_i, \quad i = 1,...,n
\intertext{where $f_i = f(x)$. We then rewrote the equation as a linear set of equations:}
- (v_{i+1} + v_{i-1} - 2v_i) = h^2f_i
\intertext{We set $h^2f_i = d_i$, and solved this equation for a few values of $i$.}
\intertext{$i = 1$:}
- (v_{1+1} + v_{1-1} - 2v_1) &= d_1 \\
- (v_2 + v_0 - 2v_1) &= d_1 \\
- v_2 - 0 + 2v_1 &= d_1
\intertext{$i = 2$:}
- (v_{2+1} + v_{2-1} - 2v_2) &= d_2 \\
- v_3 - v_1 + 2v_2) &= d_2
\intertext{$i = 3$:}
- (v_{3+1} + v_{3-1} - 2v_3) &= d_3 \\
- v_4 - v_2 + 2v_3) &= d_3	
\end{align*}
\begin{align*}
\intertext{We saw that this could be written as a linear set of equations $\vec{A} \vec{v} = \vec{d}$,}
 \left(\begin{array}{cccccc}
 	2& -1& 0 &\dots   & \dots &0 \\
     -1 & 2 & -1 &0 &\dots &\dots \\
     0&-1 &2 & -1 & 0 & \dots \\
     & \dots   & \dots &\dots   &\dots & \dots \\
     0&\dots   &  &-1 &2& -1 \\
     0&\dots    &  & 0  &-1 & 2 \\
     \end{array} \right)
\left(\begin{array}{c}
	v_1 \\
	v_2 \\
	v_3 \\
	\dots \\
	v_{n-1} \\
	v_n \\
	\end{array} \right) =
\left(\begin{array}{c}
	d_1 \\
	d_2 \\
	d_3 \\
	\dots \\
	d_{n-1} \\
	d_n \\
	\end{array} \right)
\end{align*}

\subsection{Relative error}
\noindent We computed the relative error in the data set $i = 1,...,n$ by using the expression
\begin{align*}
\epsilon_i = log_{10} \Big( \Big| \frac{v_i - u_i}{u_i}  \Big| \Big)
\end{align*}
\noindent We implemented the cosed-form solution $u(x) = 1 - (1 - e^{-10})x - e^{-10x}$ to our code and calculated the relative error when increasing $n$ to $n = 10^7$. 

\section{Results}

\begin{itemize}
\item Present results
\item Critical discussion
\item Put code etc. on GitHub and explain to reader where they can find it
\item Explanatory figures with captions, labels etc.
\end{itemize}

\subsection{Relative error}
\noindent In our python-code we took the minimum value of each list of error values. This was because the error values became negative, as a result of $u(x)$ being exponetial. We created a table of the relative error results:

\begin{center}
\begin{tabular}{| c | c |}
	\hline
	$n$ & $\epsilon$ \\
	\hline
	10 & -1.1797 \\
	$10^2$ & -3.08804 \\
	$10^3$ & -5.08005 \\
	$10^4$ & -7.07936 \\
	$10^5$ & -9.0049 \\
	$10^6$ & -6.77137 \\
	$10^7$ & -12.8074\\
	\hline 
\end{tabular}
\medskip
\\
Table 4.1 Table of the relative error $\epsilon$ for increasing $n$
\end{center}

\noindent We saw that the error became smaller when $n$ increased, but for $n = 10^6$ this was not the case. Why?

\section{Conclusion}

\begin{itemize}
\item Main findings
\item Perspectives on improvement and future work
\end{itemize}

\section{Appendix}

\begin{itemize}
\item Additional calulations
\item Selected calulations with comments
\item Code, if necessary
\item Appendix can be pushed to GitHub!
\end{itemize}

\section{References}

\begin{itemize}
\item Reference to material we based our work on(lecture notes etc.)
\item Find scientific articles, books etc.
\item BibTex - extract references online
\end{itemize}



\end{document}
